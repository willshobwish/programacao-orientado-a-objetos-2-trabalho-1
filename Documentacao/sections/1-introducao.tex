\section{Descrição do trabalho}
\subsection{Linguagem de programação utilizada}
A linguagem de programação utilizada para realização do trabalho foi Java (versão 18 do Java Development Kit).

\subsection{IDE utilizada para o desenvolvimento do trabalho}
Apache Netbeans.

\subsection{Especificações do projeto}
\textbf{Product Version:} Apache NetBeans IDE 14\\
\textbf{Java:} 18.0.1.1; Java HotSpot(TM) 64-Bit Server VM 18.0.1.1+2-6\\
\textbf{Runtime:} Java(TM) SE Runtime Environment 18.0.1.1+2-6\\

\subsection{Inicialização do projeto}
O arquivo \textit{main} do trabalho é a classe na qual é implementado a interface gráfica:\\\texttt{Trabalho01/src/InterfaceGrafico.java}.\\
Provavelmente no arquivo \texttt{build.xml} já está configurado para executar essa classe primeiramente.

\subsection{Bugs não resolvidos}
Para cadastros que envolvem data, é necessário a escrita do mês e dia com zero no inicio (por exemplo: para registrar o mês de janeiro será necessário digitar 01, caso queira registrar o dia 1, é necessário digitar 01\footnote{Isso ocorre por causa do método de registro do LocalDate utilizando o método parse, que só aceita indices iniciando em zero, provavelmente possui métodos mais corretos e menos problemáticos}).
\newpage